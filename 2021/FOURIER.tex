\begin{enumerate}[label=\thechapter.\arabic*,ref=\thechapter.\theenumi]
\item Consider the signals $x\brak{n}$=$2^{n-1} u\brak{ -n+2}$ and $y\brak{n}$=$2^{-n+2}u\brak{ n+1}$, where $u\brak{n}$ is the unit step sequence. Let $X\brak{e^{j\omega}}$ and $Y\brak{e^{j\omega}}$ be the discrete-time Fourier of $x\brak{n}$ and $y\brak{n}$,respectively. The value of the integral $\frac{1}{2\pi}\int_{0}^{2\pi} X\brak{e^{j\omega}} Y\brak{e^{-j\omega}} d \omega$
(rounded off to one decimal place) is \underline{{\hspace{1.5in}}}\\
\hfill{(GATE EC 41 2021)}\\
\solution
\iffalse
\documentclass[journal,12pt,twocolumn]{IEEEtran}
\usepackage{cite}
\usepackage{amsmath,amssymb,amsfonts,amsthm}
\usepackage{algorithmic}
\usepackage{graphicx}
\usepackage{textcomp}
\usepackage{xcolor}
\usepackage{listings}
\usepackage{enumitem}
\usepackage{mathtools}
\usepackage{gensymb}
\usepackage{comment}
\usepackage[breaklinks=true]{hyperref}
\usepackage{tkz-euclide}
\usepackage{gvv} 
\def\inputGnumericTable{} 
\usepackage[latin1]{inputenc} 
\usepackage{color} 

\newtheorem{theorem}{Theorem}[section]
\newtheorem{problem}{Problem}
\newtheorem{proposition}{Proposition}[section]
\newtheorem{lemma}{Lemma}[section]
\newtheorem{corollary}[theorem]{Corollary}
\newtheorem{example}{Example}[section]
\newtheorem{definition}[problem]{Definition}
\newcommand{\BEQA}{\begin{eqnarray}}
\newcommand{\EEQA}{\end{eqnarray}}
\newcommand{\define}{\stackrel{\triangle}{=}}
\theoremstyle{remark}
\newtheorem{rem}{Remark}

\begin{document}

\bibliographystyle{IEEEtran}
\vspace{3cm}

\title{GATE 2022-IN}
\author{EE23BTECH1205 - Avani Chouhan$^{*}$}
\maketitle
\newpage
\bigskip

\renewcommand{\thefigure}{\theenumi}
\renewcommand{\thetable}{\theenumi}

\vspace{3cm}
\textbf{Question : 18} \\
A signal \( x(t) \) is band-limited between 100 Hz and 200 Hz. A signal \( y(t) \) is related to \( x(t) \) as follows:\\

\( y(t) = x(2t - 5) \)\\
The statement that is always true is \\

\begin{enumerate}
  \item[(A)] \( y(t) \) is band-limited between 50 Hz and 100 Hz
  \item[(B)] \( y(t) \) is band-limited between 100 Hz and 200 Hz
  \item[(C)] \( y(t) \) is band-limited between 200 Hz and 400 Hz
  \item[(D)] \( y(t) \) is not band-limited 
\end{enumerate}

\hfill{(GATE IN 2022)}\\
\textbf{Solution:} \\
\fi
\begin{align}
x(t) &\rightleftharpoons X(\omega) \label{eq1}\\
x(at) &\rightleftharpoons \frac{1}{|a|} X\left(\frac{\omega}{a}\right) \label{eq2}\\
x(2t) &\rightleftharpoons \frac{1}{2} X\left(\frac{\omega}{2}\right) \label{eq3}\\
x(t - t_0) &\rightleftharpoons e^{-j\omega t_0}X(\omega) \label{eq4}\\
x(2t - 5) &\rightleftharpoons e^{-j5\omega} \cdot \frac{1}{2} X\left(\frac{\omega}{2}\right) \label{eq5}
\end{align}

The operation \(x(2t-5)\) compresses time by a factor of 2 and shifts 5 units rightward. This expands the frequency domain, doubling the bandwidth of \(x(t)\) from 100 Hz to 200 Hz to \(y(t)\) between 200 Hz and 400 Hz.\\

Hence, the correct answer is option (C).

%\end{document}


\pagebreak

\item Consider the signals \(x[n] = 2^{n-1} u[-n+2]\) and \(y[n] = 2^{-n+2} u[n+1]\), where \(u[n]\) is the unit step sequence. Let \(X(e^{j\omega})\) and \(Y(e^{j\omega})\) be the discrete-time Fourier transform of \(x[n]\) and \(y[n]\), respectively. The value of the integral
\[
\frac{1}{2\pi} \int_{0}^{2\pi} X(e^{j\omega}) Y(e^{-j\omega}) d\omega
\]
(rounded off to one decimal place) is.\\
\hfill{GATE 2021 EC 41 Q}
\solution
\iffalse
\let\negmedspace\undefined
\let\negthickspace\undefined
\documentclass[journal,12pt,twocolumn]{IEEEtran}
\usepackage{pgfplots}
\pgfplotsset{compat=1.17}
\usepackage{cite}
\usepackage{amsmath,amssymb,amsfonts,amsthm}
\usepackage{algorithmic}
\usepackage{graphicx}
\usepackage{textcomp}
\usepackage{xcolor}
\usepackage{txfonts}
\usepackage{listings}
\usepackage{enumitem}
\usepackage{mathtools}
\usepackage{gensymb}
\usepackage{comment}
\usepackage[breaklinks=true]{hyperref}
\usepackage{tkz-euclide} 
\usepackage{listings}
\usepackage{gvv}                                        
\def\inputGnumericTable{}                                 
\usepackage[latin1]{inputenc}                                
\usepackage{color}                                            
\usepackage{array}                                            
\usepackage{longtable}                                       
\usepackage{calc}                                             
\usepackage{multirow}                                         
\usepackage{hhline}                                           
\usepackage{ifthen}                                           
\usepackage{lscape}
\newtheorem{theorem}{Theorem}[section]
\newtheorem{problem}{Problem}
\newtheorem{proposition}{Proposition}[section]
\newtheorem{lemma}{Lemma}[section]
\newtheorem{corollary}[theorem]{Corollary}
\newtheorem{example}{Example}[section]
\newtheorem{definition}[problem]{Definition}
\newcommand{\BEQA}{\begin{eqnarray}}
\newcommand{\EEQA}{\end{eqnarray}}
\newcommand{\define}{\stackrel{\triangle}{=}}
\theoremstyle{remark}
\newtheorem{rem}{Remark}
\begin{document}
\bibliographystyle{IEEEtran}
\vspace{3cm}
\title{GATE EC 41Q}
\author{EE23BTECH11021 - GANNE GOPI CHANDU$^{*}$% <-this % stops a space
}
\maketitle
\bigskip
\renewcommand{\thefigure}{\theenumi}
\renewcommand{\thetable}{\theenumi}
\bibliographystyle{IEEEtran}
\textbf{Question}\\
Consider the signals \(x[n] = 2^{n-1} u[-n+2]\) and \(y[n] = 2^{-n+2} u[n+1]\), where \(u[n]\) is the unit step sequence. Let \(X(e^{j\omega})\) and \(Y(e^{j\omega})\) be the discrete-time Fourier transform of \(x[n]\) and \(y[n]\), respectively. The value of the integral
\[
\frac{1}{2\pi} \int_{0}^{2\pi} X(e^{j\omega}) Y(e^{-j\omega}) d\omega
\]
(rounded off to one decimal place) is.\\
\textbf{Solution}\\
\fi
\begin{table}[!h]
\begin{center}
\renewcommand\thetable{1}
\begin{tabular}{ |c|c|c| } 
  \hline
    Symbol & Value & description \\ 
  \hline
  $x[n] $ & $2^{n-1}u[-n+2]$ & Discrete time signal  \\ 
  \hline
  $y[n] $ & $2^{-n+2}u[n+1]$ & Discrete time signal  \\ 
  \hline
\end{tabular}
\end{center}
\caption{}
\end{table}
\begin{align}
     x[n]*y[n] && \xleftrightarrow[transform]{Fourier} && X(e^{j\omega}) Y(e^{j\omega})\\
 x[n] && \xleftrightarrow [transform]{Fourier} && X(e^{j\omega}) \\
 y[n] && \xleftrightarrow [transform]{Fourier} && Y(e^{j\omega}) 
\end{align}
The
 \begin{align}
       y(n) && \xleftrightarrow [transform]{Fourier} && y(e^{j\omega})
\end{align}
By using the time reversal property:
\begin{align}
y[-n] && \xleftrightarrow [transform]{Fourier} && y(e^{-j\omega})
\end{align}
Let assume
\begin{align}
     z[n]& = x[n] * y[-n]\\
     Z\brak{e^{j \omega}} &=X(e^{j\omega}) Y(e^{-j\omega})
 \end{align}
\begin{align}
      z[n]& =\frac{1}{2\pi} \int_{0}^{2\pi} Z(e^{j\omega})e^{j \omega n} d\omega \\
      &=\frac{1}{2\pi} \int_{0}^{2\pi}  X(e^{j\omega}) Y(e^{-j\omega})e^{j \omega n} d\omega.
 \end{align}
 putting  n=0, we get
\begin{align}
    z[0]&=\frac{1}{2\pi} \int_{0}^{2\pi} X(e^{j\omega}) Y(e^{-j\omega}) d\omega
\end{align}
\begin{align}
    z[n] = x[n] * y[-n]\\
 &= \sum_{k=-\infty}^{\infty} 2^{k-1} u[-k+2]\cdot 2^{n-k+2} u[-n+k+1]\\
 &= \sum_{k=-\infty}^{2} 2^{k-1} \cdot 2^{n-k+2} u[-n+k+1]\\
 &= \sum_{k=-\infty}^{2} 2^{k-1+n-k+2} u[-n+k+1]\\
 &= \sum_{k=-\infty}^{2} 2^{n+1} u[-n+k+1]
\end{align}

Putting $n = 0$, we get:
\begin{align}
     \frac{1}{2\pi} \int_{0}^{2\pi} X(e^{j\omega}) Y(e^{-j\omega}) d\omega &= z[0]\\
     &= \sum_{k=-\infty}^{2} 2 \cdot u[k+1] \\
     &=\sum_{k=-1}^{2} 2(1) = 2 \times 4 \\
     &= 8
\end{align}

\pagebreak
\end{enumerate}
