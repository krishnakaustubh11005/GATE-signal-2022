\begin{enumerate}[label=\thechapter.\arabic*,ref=\thechapter.\theenumi]
\item For the ordinary differential equation
\begin{align*}
\frac{d^3y}{dt^3} + 6\frac{d^2y}{dt^2} + 11\frac{dy}{dt} + 6y = 1,
\end{align*}
with initial conditions $y(0) = y'(0) = y''(0) = y'''(0) = 0$, the value of 
\begin{align*}
\lim_{{t \to \infty}} y(t) &= ?
\end{align*}
(round off to $3$ decimal places).
\hfill(GATE CH 2021)\\
\solution
\input{2021/CH/36/gate.tex}
\pagebreak
\item \textbf{Question:}
Consider the differential equation \\$\frac{d^2y}{dx^2}+8\frac{dy}{dx}+16y=0$ and the boundary conditions $y(0)=1$ and $\frac{dy}{dx}(0)=0$. The solution to equation is:\\
\hfill{(GATE.AE-1.2021)}\\
\solution
\input{2021/AE/1/gate3.tex}
\pagebreak
\item\textbf{Question:}
The solution of second-order differential equation \\ $\frac{d^2y}{dx^2}+2\frac{dy}{dx}+y=0$ with boundary conditions $y(0)=1$ and $y(1)=3$.\\
\hfill{(GATE  2021 CE.26)}\\
\solution
\input{2021/CE/26/gate4.tex}
\pagebreak
\item A system has a transfer function
\begin{align}
    G(s) = \frac{3e^{-4s}}{12s + 1}\nonumber
\end{align}
When a step-change of magnitude $M$ is given to the system input, the final value of the system output is measured to be 120. The value of M is \_\_\_\_\_.
\hfill(GATE 2021 CH Q52)\\
\solution
\input{2021/CH/52/GATE_CH_21_52.tex}
\pagebreak
\item The Bode magnitude plot for the transfer function $\frac{V_o(s)}{V_i(s)}$ of the circuit is as shown. The value of R is \_\_\_\_\_$\Omega$. \hfill(GATE 2021 EE Q20)
\begin{figure}[!ht]
    \centering
    \input{2021/EE/20/figs/tikz}
\end{figure}
\begin{figure}[!ht]
    \centering
    \includegraphics[width=\columnwidth]{2021/EE/20/figs/bode.png}
\end{figure}
\solution
\iffalse
\let\negmedspace\undefined
\let\negthickspace\undefined
\documentclass[journal,12pt,twocolumn]{IEEEtran}
\usepackage{cite}
\usepackage{amsmath,enumitem,amssymb,amsfonts,amsthm}
\usepackage{algorithmic}
\usepackage{graphicx}
\usepackage{float}
\usepackage{textcomp}
\usepackage{xcolor}
\usepackage{caption}
\usepackage{txfonts}
\usepackage{listings}
\usepackage{enumitem}
\usepackage{mathtools}
\usepackage{gensymb}
\usepackage{comment}
\usepackage[breaklinks=true]{hyperref}
\usepackage{tkz-euclide} 
\usepackage{listings}
\usepackage{tabularx}
\usepackage{gvv}                                        
\def\inputGnumericTable{}                                 
\usepackage[latin1]{inputenc}                              
\usepackage{color}                                            
\usepackage{array}                                            
\usepackage{longtable}                                       
\usepackage{calc}                                             
\usepackage{multirow}                                         
\usepackage{hhline}                                           
\usepackage{ifthen}                                        
\usepackage{lscape}
\newtheorem{theorem}{Theorem}[section]
\newtheorem{problem}{Problem}
\newtheorem{proposition}{Proposition}[section]
\newtheorem{lemma}{Lemma}[section]
\newtheorem{corollary}[theorem]{Corollary}
\newtheorem{example}{Example}[section]
\newtheorem{definition}[problem]{Definition}
\newcommand{\BEQA}{\begin{eqnarray}}
\newcommand{\EEQA}{\end{eqnarray}}
\newcommand{\define}{\stackrel{\triangle}{=}}
\theoremstyle{remark}
\newtheorem{rem}{Remark}
\usepackage{float}
\usepackage{adjustbox}
\usepackage{siunitx}
\usepackage[siunitx]{circuitikz}
\parindent 0px


\begin{document}
\bibliographystyle{IEEEtran}
\vspace{3cm}


\title{GATE: 19.2021}
\author{EE22BTECH11005- Ambati Krishna Kaustubh$^{*}$% <-this % stops a space
}

\maketitle
\newpage
\bigskip

\textbf{Question:}A single degree of freedom spring-mass-damper system is designed to ensure that the system returns to its original undisturbed position in minimum possible time without overshooting.If the mass of the system is 10kg,spring stiffness is 17400N/m and the natural frequency is 13.2rad/s,the coefficient of damping of the system is\\[2pt]

\solution
\fi 
\begin{table}[H]
    \center
    \renewcommand\thetable{1}
 

\def\arraystretch{3}
\begin{adjustbox}{width=0.5\textwidth}
    \begin{tabular}{|c|c|c|}
    \hline
        \textbf{Parameter}&\textbf{Description}&\textbf{Value}\\
        \hline
        $X(s)$&Position of mass in laplace domain&$X(s)$\\
        \hline
        $x(t)$&Position of mass in time domain&$x(t)$ \\
        \hline
        $m$ &mass of the body&10kg \\
        \hline
        $a$&coefficient of damping &$??$ \\
        \hline
        $w_d$&damping frequency of the system&$0$ \\
        \hline
       \end{tabular} 
            \end{adjustbox}
    \caption{Parameter Table}

    \label{tab:gate2021.tex}
\end{table}
The General Differential Equation of the spring-mass-Damper system is given by
\begin{align}
    m\frac{d^2x}{dt}+a\frac{dx}{dt}+kx=0
\end{align}
Taking Laplace Transform:
\begin{align}
    ms^2X(s)+asX(s)+kX(s)=0
\end{align}
\begin{align}
    \implies ms^2+as+k=0\\
    \therefore s=\frac{-a\pm \sqrt{a^2-4km}}{2m}\\
    w_d=\sqrt{a^2-4km}
\end{align}
condition for critical damping $w_d=0$
\begin{align}
    \implies a^2=4km \\
    a=834.266
\end{align}


%\end{document}

\pagebreak
\item Consider a system with transfer-function $G\brak{s}=\frac{2}{s+1}$. A unit-step function $\mu\brak{t}$ is applied to the system, which results in an output y\brak{t}. 

If $e\brak{t}=y\brak{t}-\mu\brak{t}$ then $ \lim_{t\to\infty} e(t)$ is\rule{1.5cm}{0.15mm}.
\solution
\iffalse
\let\negmedspace\undefined
\let\negthickspace\undefined
\documentclass[journal,12pt,twocolumn]{IEEEtran}
\usepackage{cite}
\usepackage{amsmath,enumitem,amssymb,amsfonts,amsthm}
\usepackage{algorithmic}
\usepackage{graphicx}
\usepackage{float}
\usepackage{textcomp}
\usepackage{xcolor}
\usepackage{caption}
\usepackage{txfonts}
\usepackage{listings}
\usepackage{enumitem}
\usepackage{mathtools}
\usepackage{gensymb}
\usepackage{comment}
\usepackage[breaklinks=true]{hyperref}
\usepackage{tkz-euclide} 
\usepackage{listings}
\usepackage{tabularx}
\usepackage{gvv}                                        
\def\inputGnumericTable{}                                 
\usepackage[latin1]{inputenc}                              
\usepackage{color}                                            
\usepackage{array}                                            
\usepackage{longtable}                                       
\usepackage{calc}                                             
\usepackage{multirow}                                         
\usepackage{hhline}                                           
\usepackage{ifthen}                                        
\usepackage{lscape}
\newtheorem{theorem}{Theorem}[section]
\newtheorem{problem}{Problem}
\newtheorem{proposition}{Proposition}[section]
\newtheorem{lemma}{Lemma}[section]
\newtheorem{corollary}[theorem]{Corollary}
\newtheorem{example}{Example}[section]
\newtheorem{definition}[problem]{Definition}
\newcommand{\BEQA}{\begin{eqnarray}}
\newcommand{\EEQA}{\end{eqnarray}}
\newcommand{\define}{\stackrel{\triangle}{=}}
\theoremstyle{remark}
\newtheorem{rem}{Remark}
\usepackage{float}
\usepackage{adjustbox}
\usepackage{siunitx}
\usepackage[siunitx]{circuitikz}
\parindent 0px


\begin{document}
\bibliographystyle{IEEEtran}
\vspace{3cm}


\title{GATE: 19.2021}
\author{EE22BTECH11005- Ambati Krishna Kaustubh$^{*}$% <-this % stops a space
}

\maketitle
\newpage
\bigskip

\textbf{Question:}A single degree of freedom spring-mass-damper system is designed to ensure that the system returns to its original undisturbed position in minimum possible time without overshooting.If the mass of the system is 10kg,spring stiffness is 17400N/m and the natural frequency is 13.2rad/s,the coefficient of damping of the system is\\[2pt]

\solution
\fi 
\begin{table}[H]
    \center
    \renewcommand\thetable{1}
 

\def\arraystretch{3}
\begin{adjustbox}{width=0.5\textwidth}
    \begin{tabular}{|c|c|c|}
    \hline
        \textbf{Parameter}&\textbf{Description}&\textbf{Value}\\
        \hline
        $X(s)$&Position of mass in laplace domain&$X(s)$\\
        \hline
        $x(t)$&Position of mass in time domain&$x(t)$ \\
        \hline
        $m$ &mass of the body&10kg \\
        \hline
        $a$&coefficient of damping &$??$ \\
        \hline
        $w_d$&damping frequency of the system&$0$ \\
        \hline
       \end{tabular} 
            \end{adjustbox}
    \caption{Parameter Table}

    \label{tab:gate2021.tex}
\end{table}
The General Differential Equation of the spring-mass-Damper system is given by
\begin{align}
    m\frac{d^2x}{dt}+a\frac{dx}{dt}+kx=0
\end{align}
Taking Laplace Transform:
\begin{align}
    ms^2X(s)+asX(s)+kX(s)=0
\end{align}
\begin{align}
    \implies ms^2+as+k=0\\
    \therefore s=\frac{-a\pm \sqrt{a^2-4km}}{2m}\\
    w_d=\sqrt{a^2-4km}
\end{align}
condition for critical damping $w_d=0$
\begin{align}
    \implies a^2=4km \\
    a=834.266
\end{align}


%\end{document}

\pagebreak
\item  Solution of differential equation $y'' + y'+ 0.25y = 0$ with initial values $y(0) = 3$ and $y'(0) = -3.5$ is
\begin{enumerate}
    \item[(A)] $ y = (3-2x)e^{0.5x} $
    \item[(B)] $ y = (3-2x)e^{-0.25x}$
    \item[(C)] $ y = (3-2x)e^{-0.5x}$
    \item[(D)] $ y = (2-3x)e^{-0.5x}$
\end{enumerate} 
\hfill(GATE AG 2021) \\
\solution
\iffalse
\let\negmedspace\undefined
\let\negthickspace\undefined
\documentclass[journal,12pt,twocolumn]{IEEEtran}
\usepackage{cite}
\usepackage{amsmath,enumitem,amssymb,amsfonts,amsthm}
\usepackage{algorithmic}
\usepackage{graphicx}
\usepackage{float}
\usepackage{textcomp}
\usepackage{xcolor}
\usepackage{caption}
\usepackage{txfonts}
\usepackage{listings}
\usepackage{enumitem}
\usepackage{mathtools}
\usepackage{gensymb}
\usepackage{comment}
\usepackage[breaklinks=true]{hyperref}
\usepackage{tkz-euclide} 
\usepackage{listings}
\usepackage{tabularx}
\usepackage{gvv}                                        
\def\inputGnumericTable{}                                 
\usepackage[latin1]{inputenc}                              
\usepackage{color}                                            
\usepackage{array}                                            
\usepackage{longtable}                                       
\usepackage{calc}                                             
\usepackage{multirow}                                         
\usepackage{hhline}                                           
\usepackage{ifthen}                                        
\usepackage{lscape}
\newtheorem{theorem}{Theorem}[section]
\newtheorem{problem}{Problem}
\newtheorem{proposition}{Proposition}[section]
\newtheorem{lemma}{Lemma}[section]
\newtheorem{corollary}[theorem]{Corollary}
\newtheorem{example}{Example}[section]
\newtheorem{definition}[problem]{Definition}
\newcommand{\BEQA}{\begin{eqnarray}}
\newcommand{\EEQA}{\end{eqnarray}}
\newcommand{\define}{\stackrel{\triangle}{=}}
\theoremstyle{remark}
\newtheorem{rem}{Remark}
\usepackage{float}
\usepackage{adjustbox}
\usepackage{siunitx}
\usepackage[siunitx]{circuitikz}
\parindent 0px


\begin{document}
\bibliographystyle{IEEEtran}
\vspace{3cm}


\title{GATE: 19.2021}
\author{EE22BTECH11005- Ambati Krishna Kaustubh$^{*}$% <-this % stops a space
}

\maketitle
\newpage
\bigskip

\textbf{Question:}A single degree of freedom spring-mass-damper system is designed to ensure that the system returns to its original undisturbed position in minimum possible time without overshooting.If the mass of the system is 10kg,spring stiffness is 17400N/m and the natural frequency is 13.2rad/s,the coefficient of damping of the system is\\[2pt]

\solution
\fi 
\begin{table}[H]
    \center
    \renewcommand\thetable{1}
 

\def\arraystretch{3}
\begin{adjustbox}{width=0.5\textwidth}
    \begin{tabular}{|c|c|c|}
    \hline
        \textbf{Parameter}&\textbf{Description}&\textbf{Value}\\
        \hline
        $X(s)$&Position of mass in laplace domain&$X(s)$\\
        \hline
        $x(t)$&Position of mass in time domain&$x(t)$ \\
        \hline
        $m$ &mass of the body&10kg \\
        \hline
        $a$&coefficient of damping &$??$ \\
        \hline
        $w_d$&damping frequency of the system&$0$ \\
        \hline
       \end{tabular} 
            \end{adjustbox}
    \caption{Parameter Table}

    \label{tab:gate2021.tex}
\end{table}
The General Differential Equation of the spring-mass-Damper system is given by
\begin{align}
    m\frac{d^2x}{dt}+a\frac{dx}{dt}+kx=0
\end{align}
Taking Laplace Transform:
\begin{align}
    ms^2X(s)+asX(s)+kX(s)=0
\end{align}
\begin{align}
    \implies ms^2+as+k=0\\
    \therefore s=\frac{-a\pm \sqrt{a^2-4km}}{2m}\\
    w_d=\sqrt{a^2-4km}
\end{align}
condition for critical damping $w_d=0$
\begin{align}
    \implies a^2=4km \\
    a=834.266
\end{align}


%\end{document}

\pagebreak
\item Consider the following first order partial differential equation, also known as the transport equation
\begin{align*}
\frac{\partial y\brak{x,t}}{\partial t}+5\frac{\partial y\brak{x,t}}{\partial x}&=0
\end{align*}
with initial conditions given by $y(x, 0) = \sin x,-\infty < x < \infty$. The value of $y(x, t)$ at $x = \pi$ and $t=\frac{\pi}{6}$ is  \rule{1cm}{0.15mm}.
\begin{enumerate}[label=(\Alph*)]
\item 1
\item 2
\item 0
\item 0.5
\end{enumerate}
\hfill(GATE 2021 BM Q28)\\
\solution
\input{2021/BM/28/28.tex}
\pagebreak
\item In the circuit, switch 'S' is in the closed position for a very long time. If the switch is opened at time $t=0$, then $i_L\brak{t}$ in amperes, for $t\geq0$ is
\input{2021/EE/29/figs/fig_1}\\
\hfill(GATE 2021 EE 29)\\
\solution
\iffalse
\let\negmedspace\undefined
\let\negthickspace\undefined
\documentclass[journal,12pt,twocolumn]{IEEEtran}
\usepackage{cite}
\usepackage{amsmath,enumitem,amssymb,amsfonts,amsthm}
\usepackage{algorithmic}
\usepackage{graphicx}
\usepackage{float}
\usepackage{textcomp}
\usepackage{xcolor}
\usepackage{caption}
\usepackage{txfonts}
\usepackage{listings}
\usepackage{enumitem}
\usepackage{mathtools}
\usepackage{gensymb}
\usepackage{comment}
\usepackage[breaklinks=true]{hyperref}
\usepackage{tkz-euclide} 
\usepackage{listings}
\usepackage{tabularx}
\usepackage{gvv}                                        
\def\inputGnumericTable{}                                 
\usepackage[latin1]{inputenc}                              
\usepackage{color}                                            
\usepackage{array}                                            
\usepackage{longtable}                                       
\usepackage{calc}                                             
\usepackage{multirow}                                         
\usepackage{hhline}                                           
\usepackage{ifthen}                                        
\usepackage{lscape}
\newtheorem{theorem}{Theorem}[section]
\newtheorem{problem}{Problem}
\newtheorem{proposition}{Proposition}[section]
\newtheorem{lemma}{Lemma}[section]
\newtheorem{corollary}[theorem]{Corollary}
\newtheorem{example}{Example}[section]
\newtheorem{definition}[problem]{Definition}
\newcommand{\BEQA}{\begin{eqnarray}}
\newcommand{\EEQA}{\end{eqnarray}}
\newcommand{\define}{\stackrel{\triangle}{=}}
\theoremstyle{remark}
\newtheorem{rem}{Remark}
\usepackage{float}
\usepackage{adjustbox}
\usepackage{siunitx}
\usepackage[siunitx]{circuitikz}
\parindent 0px


\begin{document}
\bibliographystyle{IEEEtran}
\vspace{3cm}


\title{GATE: 19.2021}
\author{EE22BTECH11005- Ambati Krishna Kaustubh$^{*}$% <-this % stops a space
}

\maketitle
\newpage
\bigskip

\textbf{Question:}A single degree of freedom spring-mass-damper system is designed to ensure that the system returns to its original undisturbed position in minimum possible time without overshooting.If the mass of the system is 10kg,spring stiffness is 17400N/m and the natural frequency is 13.2rad/s,the coefficient of damping of the system is\\[2pt]

\solution
\fi 
\begin{table}[H]
    \center
    \renewcommand\thetable{1}
 

\def\arraystretch{3}
\begin{adjustbox}{width=0.5\textwidth}
    \begin{tabular}{|c|c|c|}
    \hline
        \textbf{Parameter}&\textbf{Description}&\textbf{Value}\\
        \hline
        $X(s)$&Position of mass in laplace domain&$X(s)$\\
        \hline
        $x(t)$&Position of mass in time domain&$x(t)$ \\
        \hline
        $m$ &mass of the body&10kg \\
        \hline
        $a$&coefficient of damping &$??$ \\
        \hline
        $w_d$&damping frequency of the system&$0$ \\
        \hline
       \end{tabular} 
            \end{adjustbox}
    \caption{Parameter Table}

    \label{tab:gate2021.tex}
\end{table}
The General Differential Equation of the spring-mass-Damper system is given by
\begin{align}
    m\frac{d^2x}{dt}+a\frac{dx}{dt}+kx=0
\end{align}
Taking Laplace Transform:
\begin{align}
    ms^2X(s)+asX(s)+kX(s)=0
\end{align}
\begin{align}
    \implies ms^2+as+k=0\\
    \therefore s=\frac{-a\pm \sqrt{a^2-4km}}{2m}\\
    w_d=\sqrt{a^2-4km}
\end{align}
condition for critical damping $w_d=0$
\begin{align}
    \implies a^2=4km \\
    a=834.266
\end{align}


%\end{document}

\pagebreak
\item A single degree of freedom spring-mass-damper system is designed to ensure that the system returns to its original undisturbed position in minimum possible time without overshooting.If the mass of the system is 10kg,spring stiffness is 17400N/m and the natural frequency is 13.2rad/s,the coefficient of damping of the system is\\[2pt] \hfill{GATE-2021-AE-19}
\iffalse
\let\negmedspace\undefined
\let\negthickspace\undefined
\documentclass[journal,12pt,twocolumn]{IEEEtran}
\usepackage{cite}
\usepackage{amsmath,enumitem,amssymb,amsfonts,amsthm}
\usepackage{algorithmic}
\usepackage{graphicx}
\usepackage{float}
\usepackage{textcomp}
\usepackage{xcolor}
\usepackage{caption}
\usepackage{txfonts}
\usepackage{listings}
\usepackage{enumitem}
\usepackage{mathtools}
\usepackage{gensymb}
\usepackage{comment}
\usepackage[breaklinks=true]{hyperref}
\usepackage{tkz-euclide} 
\usepackage{listings}
\usepackage{tabularx}
\usepackage{gvv}                                        
\def\inputGnumericTable{}                                 
\usepackage[latin1]{inputenc}                              
\usepackage{color}                                            
\usepackage{array}                                            
\usepackage{longtable}                                       
\usepackage{calc}                                             
\usepackage{multirow}                                         
\usepackage{hhline}                                           
\usepackage{ifthen}                                        
\usepackage{lscape}
\newtheorem{theorem}{Theorem}[section]
\newtheorem{problem}{Problem}
\newtheorem{proposition}{Proposition}[section]
\newtheorem{lemma}{Lemma}[section]
\newtheorem{corollary}[theorem]{Corollary}
\newtheorem{example}{Example}[section]
\newtheorem{definition}[problem]{Definition}
\newcommand{\BEQA}{\begin{eqnarray}}
\newcommand{\EEQA}{\end{eqnarray}}
\newcommand{\define}{\stackrel{\triangle}{=}}
\theoremstyle{remark}
\newtheorem{rem}{Remark}
\usepackage{float}
\usepackage{adjustbox}
\usepackage{siunitx}
\usepackage[siunitx]{circuitikz}
\parindent 0px


\begin{document}
\bibliographystyle{IEEEtran}
\vspace{3cm}


\title{GATE: 19.2021}
\author{EE22BTECH11005- Ambati Krishna Kaustubh$^{*}$% <-this % stops a space
}

\maketitle
\newpage
\bigskip

\textbf{Question:}A single degree of freedom spring-mass-damper system is designed to ensure that the system returns to its original undisturbed position in minimum possible time without overshooting.If the mass of the system is 10kg,spring stiffness is 17400N/m and the natural frequency is 13.2rad/s,the coefficient of damping of the system is\\[2pt]

\solution
\fi 
\begin{table}[H]
    \center
    \renewcommand\thetable{1}
 

\def\arraystretch{3}
\begin{adjustbox}{width=0.5\textwidth}
    \begin{tabular}{|c|c|c|}
    \hline
        \textbf{Parameter}&\textbf{Description}&\textbf{Value}\\
        \hline
        $X(s)$&Position of mass in laplace domain&$X(s)$\\
        \hline
        $x(t)$&Position of mass in time domain&$x(t)$ \\
        \hline
        $m$ &mass of the body&10kg \\
        \hline
        $a$&coefficient of damping &$??$ \\
        \hline
        $w_d$&damping frequency of the system&$0$ \\
        \hline
       \end{tabular} 
            \end{adjustbox}
    \caption{Parameter Table}

    \label{tab:gate2021.tex}
\end{table}
The General Differential Equation of the spring-mass-Damper system is given by
\begin{align}
    m\frac{d^2x}{dt}+a\frac{dx}{dt}+kx=0
\end{align}
Taking Laplace Transform:
\begin{align}
    ms^2X(s)+asX(s)+kX(s)=0
\end{align}
\begin{align}
    \implies ms^2+as+k=0\\
    \therefore s=\frac{-a\pm \sqrt{a^2-4km}}{2m}\\
    w_d=\sqrt{a^2-4km}
\end{align}
condition for critical damping $w_d=0$
\begin{align}
    \implies a^2=4km \\
    a=834.266
\end{align}


%\end{document}

\pagebreak
\end{enumerate}
