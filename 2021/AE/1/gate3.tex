 \iffalse
\let\negmedspace\undefined
\let\negthickspace\undefined
\documentclass[journal,12pt,twocolumn]{IEEEtran}
\usepackage{xparse}
\usepackage{cite}
\usepackage{amsmath,amssymb,amsfonts,amsthm}
\usepackage{algorithmic}
\usepackage{graphicx}
\usepackage{textcomp}
\usepackage{xcolor}
\usepackage{txfonts}
\usepackage{listings}
\usepackage{enumitem}
\usepackage{mathtools}
\usepackage{gensymb}
\usepackage{comment}
\usepackage[breaklinks=true]{hyperref}
\usepackage{tkz-euclide}
\usepackage{listings}
\usepackage{gvv}
\def\inputGnumericTable{}
\usepackage[latin1]{inputenc}
\usepackage{color}
\usepackage{array}
\usepackage{longtable}
\usepackage{calc}
\usepackage{multirow}
\usepackage{hhline}
\usepackage{ifthen}
\usepackage{lscape}
\begin{enumerate}[label=\thechapter.\arabic*,ref=\thechapter.\theenumi]
\numberwithin{equation}{enumi}
\numberwithin{figure}{enumi}
\numberwithin{table}{enumi}
\item Laplace Transform of Partial Differentials\\
Let a function $y\brak{x,t}$ be defined for all $t>0$ and assumed to be bounded. Appling Laplace transform in t considering x as a parameter,
\begin{align}
 \mathcal{L}\brak{y\brak{x,t}} &= \int_{0}^{\infty}e^{-st}y\brak{x,t}dt\\
 &= Y\brak{x,s}
\end{align}
Let $\dfrac{\partial y\brak{x,t}}{\partial t}$ be $y_t\brak{x,t}$ and $\dfrac{\partial y\brak{x,t}}{\partial x}$ be $y_x\brak{x,t}$, then
\begin{align}
 \mathcal{L}\brak{y_t\brak{x,t}} &= \int_{0}^{\infty}e^{-st}y_t\brak{x,t}dt\\
 &= \left. e^{-st}y\brak{x,t}\right|_{0}^{\infty} + s\int_{0}^{\infty}e^{-st} y\brak{x,t} dt\\
 &= sY\brak{x,s} - y\brak{x,0} \label{L(y_t(x,t))}\\
 \mathcal{L}\brak{y_x\brak{x,t}} &= \int_{0}^{\infty}e^{-st}y_x\brak{x,t}dt\\
 &= \dfrac{d}{dx}\int_{0}^{\infty}e^{-st}y\brak{x,t}dt \label{L(y_x(x,t))}\\
 &= \dfrac{dY\brak{x,s}}{dx}
\end{align}
\item Laplace transform of f(t):
\begin{align}
        f(t)u(t)\system{L}&\int_{0}^{\infty}f(t)e^{-st}\; dt\\
        &=F(s)\label{lap transform}
\end{align}
\item Laplace transform of powers of $t$\\
        Let $f(t)=t^nu(t)$\\
From \eqref{lap transform},and considering $h=st$
\begin{align}
        F(s)&=\frac{1}{s^{n+1}}\int_{0}^{\infty}h^ne^{-h}\;dh\label{lap 1}\\
        (n-1)!&=\int_0^\infty e^{-t}t^{n-1}\;dt\text{ (Gamma function)}\label{gamma}
\end{align}
From \eqref{lap 1},\eqref{gamma}
\begin{align}
        F(s)&=\frac{n!}{s^{n+1}}\\
         t^nu(t)&\system{L}\frac{n!}{s^{n+1}}\label{lap exp}
\end{align}
\item Frequency shift property:\\
        Let $f(t)=y(t)e^{-at}u(t)$\\
From\eqref{lap transform},
\begin{align}
        F(s)&=\int_0^{\infty}y(t)e^{-(s+a)t}\;dt\\
         y(t)e^{-at}u(t)&\system{L}Y(s+a)\label{lap freq shift}
\end{align}
\item Inverse Laplace for partial fractions\\
From \eqref{lap exp},\eqref{lap freq shift} we get
\begin{align}
    &\frac{b}{(s+a)^n}\xleftrightarrow{\mathcal{L}^{-1}}\frac{b}{(n-1)!}\cdot t^{n-1} e^{-at}\cdot u(t)\label{inv lap (partial fractions)}
\end{align}
\item Laplace transform of derivatives:\\
        Let $f(t)=y'(t)u(t)$\\
From \eqref{lap transform}, integration by parts,recursion
\begin{align}
        F(s)&=\int_{0}^\infty e^{-st}\; dy\\
        &=[y(t)e^{-st}]_0^\infty+s\int_0^\infty y(t)e^{-st}dt\\
        &=-y(0)+sY(s)\label{lap 2}
\end{align}
From\eqref{lap 2},recursion
\begin{align}
        y'(t)u(t)&\system{L}sY(s)-\int y'(t)\;dt\vert_{t=0}\\
        y^{(n)}(t)u(t)&\system{L}s^nY(s)-\sum\limits_{k=0}^{n-1}s^{(n-1-k)}y^{(k)}(0)\label{lap (derivatives)}
\end{align}
\end{enumerate}

\newtheorem{theorem}{Theorem}[section]
\newtheorem{problem}{Problem}
\newtheorem{proposition}{Proposition}[section]
\newtheorem{lemma}{Lemma}[section]
\newtheorem{corollary}[theorem]{Corollary}
\newtheorem{example}{Example}[section]
\newtheorem{definition}[problem]{Definition}
\newcommand{\BEQA}{\begin{eqnarray}}
\newcommand{\EEQA}{\end{eqnarray}}
\newcommand{\define}{\stackrel{\triangle}{=}}
\theoremstyle{remark}
\newtheorem{rem}{Remark}
\begin{document}

\bibliographystyle{IEEEtran}
\vspace{3cm}

\title{GATE-AE.1}
\author{EE23BTECH11046 - Poluri Hemanth$^{*}$}
\maketitle
\textbf{Question:}
Consider the differential equation \\$\frac{d^2y}{dx^2}+8\frac{dy}{dx}+16y=0$ and the boundary conditions $y(0)=1$ and $\frac{dy}{dx}(0)=0$. The solution to equation is:\\
\hfill{(GATE.AE-1.2021)}\\
\textbf{Solution:}\\
\fi
\begin{table}[h!]
    % Change address in github
        \begin{table}[ht!]
\centering
\begin{tabular}{ |c|c|c| } 
 \hline
Symbols & Description & Values  \\
\hline
 $V_s$ & Input voltage & 220 V and 50Hz\\
 \hline
 $\chi_L$ & Impedance across inductor & $j\omega L$\\
 \hline
 $\chi_C$ & Impedance across capacitor & $\frac{-j}{\omega C}$\\
 \hline
 $Z$& Impedance across the entire circuit & $\frac{1}{R+j\omega L +\frac{-j}{\omega C}}$\\
 \hline
\end{tabular}
\caption{Parameters, Descriptions, and Values}
\label{table:ee25-bm54-gate2022}
\end{table}




        \caption{Parameters}
\end{table}\\
From \ref{lap (derivatives)}
\begin{align}
	\frac{d^2y}{dx^2}+8\frac{dy}{dx}+16y&\Large\xleftrightarrow{\mathcal{L}}s^2Y(s)-sy(0)-y'(0)+8sY(s)-8y(0)+16Y(s)
\end{align}
\begin{align}
	Y(s)(s^2+8s+16)&=s+8
\end{align}
\begin{align}
	\Rightarrow Y(s)&=\frac{s+8}{s^2+8s+16}\\
	&=\frac{1}{s+4}+\frac{4}{(s+4)^2}\label{1ae.1}
\end{align}
For inversion of $Y(s)$ in partial fractions-
\\ From \ref{inv lap (partial fractions)}
\begin{align}
	&\frac{b}{(s+a)^n}\Large\xleftrightarrow{\mathcal{L}^{-1}}\frac{b}{(n-1)!}\cdot x^{n-1} e^{-ax}\cdot u(x)\label{invae.1}
\end{align}
\\
Applying Laplace inverse-\\
\\From \eqref{1ae.1},\eqref{invae.1}
\begin{align}
	y(x)&=\frac{1}{0!} e^{-4x}\cdot u(x)+\frac{4}{1!}x\cdot e^{-4x}\cdot u(x)\\
	&=(1+4x)e^{-4x}u(x)
\end{align}
\newpage
\begin{figure}[h!]
    \centering
    \includegraphics[width=1\linewidth]{2021/AE/1/figures/figure.png}
        \caption{Plot of y(x)}
\end{figure}


%\end{document}





