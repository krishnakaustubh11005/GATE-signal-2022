 \iffalse
\let\negmedspace\undefined
\let\negthickspace\undefined
\documentclass[journal,12pt,twocolumn]{IEEEtran}
\usepackage{cite}
\usepackage{amsmath,amssymb,amsfonts,amsthm}
\usepackage{algorithmic}
\usepackage{graphicx}
\usepackage{textcomp}
\usepackage{xcolor}
\usepackage{txfonts}
\usepackage{listings}
\usepackage{enumitem}
\usepackage{mathtools}
\usepackage{gensymb}
\usepackage{comment}
\usepackage[breaklinks=true]{hyperref}
\usepackage{tkz-euclide} 
\usepackage{tikz}
% \usetikzlibrary{positioning, arrows.meta}
\usepackage{listings}
\usepackage{gvv} 
\usepackage{caption}
\def\inputGnumericTable{}                   

%\usepackage[latin1]{inputenc}                                
\usepackage{color}                                            
\usepackage{array}                                            
\usepackage{longtable}                                       
\usepackage{calc}                                             
\usepackage{multirow}                                         
\usepackage{hhline}                                           
\usepackage{ifthen}                                           
\usepackage{lscape}
\usepackage{tikz}
\newtheorem{theorem}{Theorem}[section]
\newtheorem{problem}{Problem}
\newtheorem{proposition}{Proposition}[section]
\newtheorem{lemma}{Lemma}[section]
\newtheorem{corollary}[theorem]{Corollary}
\newtheorem{example}{Example}[section]
\newtheorem{definition}[problem]{Definition}
\newcommand{\BEQA}{\begin{eqnarray}}
\newcommand{\EEQA}{\end{eqnarray}}
\newcommand{\define}{\stackrel{\triangle}{=}}
\theoremstyle{remark}
\newtheorem{rem}{Remark}

\begin{document}

\bibliographystyle{IEEEtran}
\vspace{3cm}

\title{GATE: EE - 31.2021}
\author{EE23BTECH11013 - Avyaaz$^{*}$% <-this % stops a space 
}
\maketitle
\newpage
\bigskip

\renewcommand{\thefigure}{\arabic{figure}}
\renewcommand{\thetable}{\arabic{table}}

\large\textbf{\textsl{Question:}}
The causal signal with Z transform $z^2(z - a)^{-2}$ is ($u(n)$ is unit step signal)
\begin{enumerate}
    \item $a^{2n}u(n)$
    \item $(n + 1)a^nu(n)$
    \item $n^{-1}a^nu(n)$
    \item $n^2a^nu(n)$
\end{enumerate}

\hfill(GATE EE 2021) \\
\solution
\fi
% \begin{table}[htbp]
%     \centering
%      \begin{table}[!ht] 
\centering
\setlength{\extrarowheight}{8pt}
\begin{tabular}{|l|l|l|}
    \hline
    \textbf{Parameter} & \textbf{Description} & \textbf{Values}\\
    \hline
     m & load of system &  \\
    \hline
     k & stiffness of system &  \\
    \hline
     $\omega_n$ & Natural frequency & $\sqrt{\frac{k}{m}}$ \\
    \hline
    $\zeta$ & Damping ratio & $\frac{c}{2m\omega_n}$ \\
    \hline
     y\brak{t} & Output of system & \\
    \hline
     x\brak{t} & Input to the system & \\
    \hline
     c & Damping coefficient & \\
    \hline
    T\brak{s} & Transfer function of system & $\frac{Y\brak{s}}{R\brak{s}}$\\
    \hline
  \end{tabular}
  \vspace{4mm}
 \caption{Parameter Table}
 \label{tab:table0_ee_22_39}
\end{table}

%     \caption{}
%     \label{tab:my_label.41.IN.2022}
% \end{table}

% \begin{figure}[!htbp]
%     \resizebox{0.501\textwidth}{!}{\tikzset{
    block/.style = {draw, fill=white, rectangle, minimum height=3em, minimum width=3em},
    tmp/.style  = {coordinate}, 
    minus/.style= {draw, fill=white, circle, node distance=1cm, append after command={\pgfextra \draw ($(\tikzlastnode.center) + (-0.15,0)$) -- ($(\tikzlastnode.center) + (0.15,0)$) node[above] {$-$}; \endpgfextra}},
    plus/.style= {draw, fill=white, circle, node distance=1cm, append after command={\pgfextra \draw ($(\tikzlastnode.center) + (-0.15,0)$) -- ($(\tikzlastnode.center) + (0.15,0)$) node[above] {$+$}; \endpgfextra}},
    input/.style = {coordinate},
    output/.style= {coordinate},
    pinstyle/.style = {pin edge={to-,thin,black}}
}


\begin{tikzpicture}[auto, node distance=2cm]
    % Blocks
    \node [input, name=rinput] (rinput) at (0,0) {};
    \node [minus] (sum1) at (1,0) {};
    \node [block] (controller) at (3,0) {$K_{p}$};
    \node [block] (kd) at (3,-2) {$sK_D$};
    \node [block] (up) at (3,2) {\Large$\frac{K_{i}}{s}$};
    \node [plus] (sum2) at (5,0) {};
    \node [block] (system) at (7,0) {$P\brak{s}=\frac{1}{\brak{s+1}\brak{s+2}}$};
    \node [output] (output) at (9,0) {};
    \node [tmp] (tmp1) at (3,-4) {$H(s)$};

    % Connectors
    \draw [->] (rinput) -- node[below]{$r\brak{t}$} (sum1);
    \draw [->] (sum1) -- node[name=z,anchor=north,fill=white,circle,inner sep=1pt]{$e\brak{t}$} (controller);
    \draw [->] (controller) -- (sum2);
    \draw [->] (sum2) -- node[above, pos=-2.5]{$G_c\brak{t}$} (system);
    \draw [->] (system) -- node [pos = 1,name=y] {$L\brak{t}$} (output);
    \draw [->] (z) |- (up);
    \draw [->] (up) -| (sum2);
    \draw [->] (z) |- (kd);
    \draw [->] (kd) -| (sum2);
    \draw [->] (y) |- (tmp1);
    \draw [->] (tmp1) -| (sum1);
\end{tikzpicture}
}
%     \caption{Block Diagram of System}
%     \label{fig:gate_IN_Q41_blockdiagram}
% \end{figure}


Z-transform of a causal signal is, 
\begin{align}
    X(z) = z^2(z - a)^{-2} = \frac{1}{(1 - az^{-1})^2};|z| > |a|\label{eq:given.EE.31.2021}
\end{align}
The Z transform pair for $a^nu(n)$ signal is given by :
\begin{align}
    a^nu(n) \longleftrightarrow \frac{1}{1 - az^{-1}}
\end{align}
Using differentiation in z-domain property:
\begin{align}
    na^nu(n) &\longleftrightarrow -z\frac{d}{dz}\left(\frac{1}{1 - az^{-1}}\right) \\
     \implies    na^nu(n) &\longleftrightarrow \frac{az^{-1}}{(1 - az^{-1})^2}
\end{align}
Using time-shifting property:
\begin{align}
  (n + 1)a^{n + 1}u(n + 1) \longleftrightarrow \frac{az^{-1}}{(1 - az^{-1})^2}z\\
  (n + 1)a^nu(n + 1) \longleftrightarrow \frac{1}{(1 - az^{-1})^2}\label{EQ:TIME.31.EE.2021}
\end{align}
From \eqref{eq:given.EE.31.2021} and \eqref{EQ:TIME.31.EE.2021}, Inverse Z transform is :
\begin{align}
    x(n) = (n + 1)a^nu(n + 1)
\end{align}
Sequence \(u(n + 1)\) exist for\(-1 \leq n < \infty\), but the factor \((n + 1)\) is zero for \(n = -1\), so \(x(n)\) may be expressed as a causal sequence. 
\begin{align}
    x(n) = (n + 1)a^nu(n)
\end{align}



\begin{figure}[htbp]
    \centering
    \includegraphics[width = \columnwidth]{2021/EE/31/figs/transform.png}
	\caption{$x(n) vs n $ using $a = 1.5$}
    \label{fig:graph1.41.IN.2022}
\end{figure}

