\iffalse
\documentclass[journal,12pt,twocolumn]{IEEEtran}
\usepackage{amsmath,amssymb,amsfonts,amsthm}
\usepackage{txfonts}
\usepackage{tkz-euclide}
\usepackage{listings}
\usepackage{gvv}
\usepackage[latin1]{inputenc}
\usepackage{adjustbox}
\usepackage{array}
\usepackage{tabularx}
\usepackage{pgf}
\usepackage{lmodern}
\usepackage{circuitikz}
\usepackage{tikz}
\usepackage{graphicx}
\usepackage[english]{babel}

\begin{document}
\bibliographystyle{IEEEtran}

\vspace{3cm}

\title{}
\author{EE23BTECH11047 - Deepakreddy P
}
\maketitle
\newpage
\bigskip

\noindent \textbf{32} \quad An analog signal is sampled at 100 MHz to generate 1024 samples. Only
these samples are used to evaluate 1024-point FFT. The separation between
adjacent frequency points ($\Delta$F) in FFT is \rule{1cm}{0.5mm} kHz.\\
\hfill (GATE BM 2021)\\
\solution
\fi

\begin{center}
    \begin{table}[ht]
        \begin{table}[!ht] 
\centering
\setlength{\extrarowheight}{8pt}
\begin{tabular}{|l|l|l|}
    \hline
    \textbf{Parameter} & \textbf{Description} & \textbf{Values}\\
    \hline
     m & load of system &  \\
    \hline
     k & stiffness of system &  \\
    \hline
     $\omega_n$ & Natural frequency & $\sqrt{\frac{k}{m}}$ \\
    \hline
    $\zeta$ & Damping ratio & $\frac{c}{2m\omega_n}$ \\
    \hline
     y\brak{t} & Output of system & \\
    \hline
     x\brak{t} & Input to the system & \\
    \hline
     c & Damping coefficient & \\
    \hline
    T\brak{s} & Transfer function of system & $\frac{Y\brak{s}}{R\brak{s}}$\\
    \hline
  \end{tabular}
  \vspace{4mm}
 \caption{Parameter Table}
 \label{tab:table0_ee_22_39}
\end{table}

    \end{table}
\end{center}

\begin{align}
    \Delta F &= \frac{f_s}{N}\\
    \Delta F &= \frac{100}{1024} MHz\\
    \Delta F &= \frac{10^5}{1024} kHz\\
    \Delta F &= 97.66kHz
\end{align}


\begin{figure}[ht]
   \centering
   \includegraphics[width=1.1\columnwidth]{2021/BM/32/figs/fig1.png}
   \caption{Time Domain Signal}
\end{figure}

\begin{figure}[ht]
   \centering
   \includegraphics[width=1.1\columnwidth]{2021/BM/32/figs/fig2.png}
   \caption{Frequency Spectrum}
\end{figure}






%\end{document}

