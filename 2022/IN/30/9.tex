\iffalse
\let\negmedspace\undefined
\let\negthickspace\undefined
\documentclass[a4,12pt,onecolumn]{IEEEtran}
\usepackage{amsmath,amssymb,amsfonts,amsthm}
\usepackage{algorithmic}
\usepackage{graphicx}
\usepackage{textcomp}
\usepackage{xcolor}
\usepackage{txfonts}
\usepackage{listings}
\usepackage{enumitem}
\usepackage{mathtools}
\usepackage{gensymb}
\usepackage[breaklinks=true]{hyperref}
\usepackage{tkz-euclide}
\usepackage{listings}
\usepackage{circuitikz}
\usepackage{gvv}
\newcommand{\mybmat}[1]{\ensuremath{\begin{bmatrix}#1\end{bmatrix}}}
\begin{document}
\title{
\Huge\textbf{ GATE 2022 Assignment}\\
\Huge\textbf{EE1205} Signals and Systems\\
}
\large\author{Kurre Vinay\\EE23BTECH11036}
\maketitle
\textbf{Question:}
The transfer function of a system is:\\
\begin{center}
$\displaystyle \frac { \brak{s+1}\brak{s+3}}{\brak{s+5}\brak{s+7}\brak{s+9}}$\\
\end{center}
In the state-space representation of the system, the minimum number of state variables (in integer) necessary is\underline{\hspace{1cm}}.\\
\hfill(GATE IN 2022)\\
\solution\\
\fi
\begin{table}[ht!]
\begin{center}
\begin{tabular}{|c|c|c|}
	   \hline
	   variable&value&description\\
	   \hline
	   U\brak{s}&-&input function of the system\\
	   \hline
	   Y\brak{s}&-&output function of the system\\
	   \hline
	   H\brak{s}&$\frac { \brak{s+1}\brak{s+3}}{\brak{s+5}\brak{s+7}\brak{s+9}}$&transfer function of the system.\\
	   \hline
	   I&-&identity matrix \\
	   \hline
	   $\vec{\dot{x}}\brak{t}$ & $A\vec{x}\brak{t} + Bu\brak{t}$&derivative of State function of $ \vec{x}\brak{t}$\\
	   \hline
\end{tabular}
\caption{Table: Input Parameters}
\label{tab:1}
\end{center}
\end{table}
From \tabref{tab:1}
\begin{align}
H\brak{s}&=\frac{\brak{s+1}\brak{s+3}}{\brak{s+5}\brak{s+7}\brak{s+9}}\\
H\brak{s}&= \frac{P}{s+5} +\frac{Q}{s+7} + \frac{R}{s+9}\\
\brak{s+1}\brak{s+3}&=P\brak{s+7}\brak{s+9}+Q\brak{s+5}\brak{s+9}+R\brak{s+5}\brak{s+7} \label{eq:gate2022}
\end{align}
By solving equation \eqref{eq:gate2022} , we get\\
\begin{center}
P = 1\\
Q = -6\\
R = 6
\end{center}

\begin{align}
\implies H\brak{s}&=\frac{1}{s+5} -\frac{6}{s+7} + \frac{6}{s+9}\\
\end{align}
The state-space representation of the system is given by:
\begin{align}
\vec{\dot{x}}\brak{t} &=A\vec{x}\brak{t} + Bu\brak{t}\\
\vec{y}\brak{t} &= C\vec{x}\brak{t} + Du\brak{t}\\
H\brak{s}&=\frac{Y\brak{s}}{U\brak{s}}=C\myvec {sI-A}^{-1}B + D  
\end{align}
Comparing the coefficients:

\begin{align}
A &= \text{coefficient of } s \text{ in } \brak{sI-A}^{-1} \\
B &= \text{coefficient of }  U\brak{s}\\
C &= \text{coefficient of } Y\brak{s} \\
D &= \text{constant term}
\end{align}
 The denominator $\brak{s+5}\brak{s+7}\brak{s+9}$ suggests that the system has three poles. Thus, we'll have a third-order state-space model, and A will be a $3\times 3$matrix.
\begin{align}
\brak{s+5}\brak{s+7}\brak{s+9} &=s^3+21s^2+143s+315\\
A &=  \myvec{0 & 1&0 \\ 0 & 0&1\\-21&-143&-315}\\
\end{align}
A is a $ 3\times 3$ matrix, then the characteristic polynomial will have a degree equal to the size of A, which is $3$.\\
Therefore, the system order, and hence the minimum number of state variables, will be 3.\\
