\documentclass[journal,12pt,onecolumn]{IEEEtran}
\usepackage{cite}
\usepackage{amsmath,amssymb,amsfonts,amsthm}
\usepackage{algorithmic}
\usepackage{graphicx}
\usepackage{textcomp}
\usepackage{xcolor}
\usepackage{txfonts}
\usepackage{listings}
\usepackage{enumitem}
\usepackage{mathtools}
\usepackage{gensymb}
\usepackage{comment}
\usepackage[breaklinks=true]{hyperref}
\usepackage{tkz-euclide}
\usepackage{listings}
\usepackage{gvv}
\def\inputGnumericTable{}
\usepackage[latin1]{inputenc}
\usepackage{color}
\usepackage{array}
\usepackage{longtable}
\usepackage{calc}
\usepackage{multirow}
\usepackage{hhline}
\usepackage{ifthen}
\usepackage{lscape}

\newtheorem{theorem}{Theorem}[section]
\newtheorem{problem}{Problem}
\newtheorem{proposition}{Proposition}[section]
\newtheorem{lemma}{Lemma}[section]
\newtheorem{corollary}[theorem]{Corollary}
\newtheorem{example}{Example}[section]
\newtheorem{definition}[problem]{Definition}
\newcommand{\BEQA}{\begin{eqnarray}}
    \newcommand{\EEQA}{\end{eqnarray}}
\newcommand{\define}{\stackrel{\triangle}{=}}
\theoremstyle{remark}
\newtheorem{rem}{Remark}

\begin{document}
    
    \bibliographystyle{IEEEtran}
    \vspace{3cm}
    
    \title{Gate 2022 EC Q55}
    \author{EE23BTECH11212 - Manugunta Meghana Sai$^{*}$% <-this % stops a space
    }
    \maketitle
    \bigskip
    
    \renewcommand{\thefigure}{\theenumi}
    \renewcommand{\thetable}{\theenumi}
    
    \vspace{3cm}
    \textbf{Gate 2022 EE Q55} 
    
    For a vector $\bar{x} = [x[0], x[1], \dots, x[7] ]$, the $8$-point discrete Fourier transform (DFT) is denoted by $\bar{X} = \text{DFT}(\bar{x}) = [X[0],X[1],\dots,X[7]]$, where
    \begin{align*}
    X[k] = \sum_{n=0}^{7}x[n]\exp\left(-j\frac{2\pi}{8}nk\right).
    \end{align*} 
    Here $j = \sqrt{-1}$. If $\bar{x} = [1,0,0,0,2,0,0,0]$ and $\bar{y} = \text{DFT}(\text{DFT}(\bar{x}))$, then the value of $y[0]$ is\\
    \solution
    \begin{table}[h!]
 	\centering
 	\resizebox{6 cm}{!}{
 		\begin{table}[!ht] 
\centering
\setlength{\extrarowheight}{8pt}
\begin{tabular}{|l|l|l|}
    \hline
    \textbf{Parameter} & \textbf{Description} & \textbf{Values}\\
    \hline
     m & load of system &  \\
    \hline
     k & stiffness of system &  \\
    \hline
     $\omega_n$ & Natural frequency & $\sqrt{\frac{k}{m}}$ \\
    \hline
    $\zeta$ & Damping ratio & $\frac{c}{2m\omega_n}$ \\
    \hline
     y\brak{t} & Output of system & \\
    \hline
     x\brak{t} & Input to the system & \\
    \hline
     c & Damping coefficient & \\
    \hline
    T\brak{s} & Transfer function of system & $\frac{Y\brak{s}}{R\brak{s}}$\\
    \hline
  \end{tabular}
  \vspace{4mm}
 \caption{Parameter Table}
 \label{tab:table0_ee_22_39}
\end{table}

 	}
 	\caption{Given Parameters}
 	\label{tab:msmECgate55tab1} 
 \end{table} 
    \\DFT of $\bar{x}$
    \begin{align}
    X[k] = \sum_{n=0}^{7}x[n]\exp\left(-j\frac{2\pi}{8}nk\right)
    \end{align}
    As the only non-zero values in x are x[0] and x[4]:
    \begin{align}
    X[k] = x[0] + x[4]\exp\left(-j\pi k\right)
    \end{align}
    After substituting the values of k ranging from $0$ to $7$,
    \begin{align}
    \bar{X} &= \text{DFT}(\bar{x}) = [X[0],X[1],\dots,X[7]]\\
    \bar{X} &= [3,-1,3,-1,3,-1,3,-1]
    \end{align}
    \begin{align}
    \bar{y} &= \text{DFT}(\text{DFT}(\bar{x}))\\
    \bar{y} &= [3,-1,3,-1,3,-1,3,-1]\\
    y[0] &= \sum_{n=0}^{7}x[n]\\
    &= x[0] + x[1] + \dots + x[7]\\
    &= 3 -1 +3 -1 +3 -1 +3 -1 = 8
    \end{align}
\end{document}

